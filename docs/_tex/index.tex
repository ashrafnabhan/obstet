% Options for packages loaded elsewhere
% Options for packages loaded elsewhere
\PassOptionsToPackage{unicode}{hyperref}
\PassOptionsToPackage{hyphens}{url}
\PassOptionsToPackage{dvipsnames,svgnames,x11names}{xcolor}
%
\documentclass[
  letterpaper,
  DIV=11,
  numbers=noendperiod]{scrartcl}
\usepackage{xcolor}
\usepackage{amsmath,amssymb}
\setcounter{secnumdepth}{-\maxdimen} % remove section numbering
\usepackage{iftex}
\ifPDFTeX
  \usepackage[T1]{fontenc}
  \usepackage[utf8]{inputenc}
  \usepackage{textcomp} % provide euro and other symbols
\else % if luatex or xetex
  \usepackage{unicode-math} % this also loads fontspec
  \defaultfontfeatures{Scale=MatchLowercase}
  \defaultfontfeatures[\rmfamily]{Ligatures=TeX,Scale=1}
\fi
\usepackage{lmodern}
\ifPDFTeX\else
  % xetex/luatex font selection
\fi
% Use upquote if available, for straight quotes in verbatim environments
\IfFileExists{upquote.sty}{\usepackage{upquote}}{}
\IfFileExists{microtype.sty}{% use microtype if available
  \usepackage[]{microtype}
  \UseMicrotypeSet[protrusion]{basicmath} % disable protrusion for tt fonts
}{}
\makeatletter
\@ifundefined{KOMAClassName}{% if non-KOMA class
  \IfFileExists{parskip.sty}{%
    \usepackage{parskip}
  }{% else
    \setlength{\parindent}{0pt}
    \setlength{\parskip}{6pt plus 2pt minus 1pt}}
}{% if KOMA class
  \KOMAoptions{parskip=half}}
\makeatother
% Make \paragraph and \subparagraph free-standing
\makeatletter
\ifx\paragraph\undefined\else
  \let\oldparagraph\paragraph
  \renewcommand{\paragraph}{
    \@ifstar
      \xxxParagraphStar
      \xxxParagraphNoStar
  }
  \newcommand{\xxxParagraphStar}[1]{\oldparagraph*{#1}\mbox{}}
  \newcommand{\xxxParagraphNoStar}[1]{\oldparagraph{#1}\mbox{}}
\fi
\ifx\subparagraph\undefined\else
  \let\oldsubparagraph\subparagraph
  \renewcommand{\subparagraph}{
    \@ifstar
      \xxxSubParagraphStar
      \xxxSubParagraphNoStar
  }
  \newcommand{\xxxSubParagraphStar}[1]{\oldsubparagraph*{#1}\mbox{}}
  \newcommand{\xxxSubParagraphNoStar}[1]{\oldsubparagraph{#1}\mbox{}}
\fi
\makeatother


\usepackage{longtable,booktabs,array}
\usepackage{calc} % for calculating minipage widths
% Correct order of tables after \paragraph or \subparagraph
\usepackage{etoolbox}
\makeatletter
\patchcmd\longtable{\par}{\if@noskipsec\mbox{}\fi\par}{}{}
\makeatother
% Allow footnotes in longtable head/foot
\IfFileExists{footnotehyper.sty}{\usepackage{footnotehyper}}{\usepackage{footnote}}
\makesavenoteenv{longtable}
\usepackage{graphicx}
\makeatletter
\newsavebox\pandoc@box
\newcommand*\pandocbounded[1]{% scales image to fit in text height/width
  \sbox\pandoc@box{#1}%
  \Gscale@div\@tempa{\textheight}{\dimexpr\ht\pandoc@box+\dp\pandoc@box\relax}%
  \Gscale@div\@tempb{\linewidth}{\wd\pandoc@box}%
  \ifdim\@tempb\p@<\@tempa\p@\let\@tempa\@tempb\fi% select the smaller of both
  \ifdim\@tempa\p@<\p@\scalebox{\@tempa}{\usebox\pandoc@box}%
  \else\usebox{\pandoc@box}%
  \fi%
}
% Set default figure placement to htbp
\def\fps@figure{htbp}
\makeatother


% definitions for citeproc citations
\NewDocumentCommand\citeproctext{}{}
\NewDocumentCommand\citeproc{mm}{%
  \begingroup\def\citeproctext{#2}\cite{#1}\endgroup}
\makeatletter
 % allow citations to break across lines
 \let\@cite@ofmt\@firstofone
 % avoid brackets around text for \cite:
 \def\@biblabel#1{}
 \def\@cite#1#2{{#1\if@tempswa , #2\fi}}
\makeatother
\newlength{\cslhangindent}
\setlength{\cslhangindent}{1.5em}
\newlength{\csllabelwidth}
\setlength{\csllabelwidth}{3em}
\newenvironment{CSLReferences}[2] % #1 hanging-indent, #2 entry-spacing
 {\begin{list}{}{%
  \setlength{\itemindent}{0pt}
  \setlength{\leftmargin}{0pt}
  \setlength{\parsep}{0pt}
  % turn on hanging indent if param 1 is 1
  \ifodd #1
   \setlength{\leftmargin}{\cslhangindent}
   \setlength{\itemindent}{-1\cslhangindent}
  \fi
  % set entry spacing
  \setlength{\itemsep}{#2\baselineskip}}}
 {\end{list}}
\usepackage{calc}
\newcommand{\CSLBlock}[1]{\hfill\break\parbox[t]{\linewidth}{\strut\ignorespaces#1\strut}}
\newcommand{\CSLLeftMargin}[1]{\parbox[t]{\csllabelwidth}{\strut#1\strut}}
\newcommand{\CSLRightInline}[1]{\parbox[t]{\linewidth - \csllabelwidth}{\strut#1\strut}}
\newcommand{\CSLIndent}[1]{\hspace{\cslhangindent}#1}



\setlength{\emergencystretch}{3em} % prevent overfull lines

\providecommand{\tightlist}{%
  \setlength{\itemsep}{0pt}\setlength{\parskip}{0pt}}



 


\KOMAoption{captions}{tableheading}
\makeatletter
\@ifpackageloaded{caption}{}{\usepackage{caption}}
\AtBeginDocument{%
\ifdefined\contentsname
  \renewcommand*\contentsname{Table of contents}
\else
  \newcommand\contentsname{Table of contents}
\fi
\ifdefined\listfigurename
  \renewcommand*\listfigurename{List of Figures}
\else
  \newcommand\listfigurename{List of Figures}
\fi
\ifdefined\listtablename
  \renewcommand*\listtablename{List of Tables}
\else
  \newcommand\listtablename{List of Tables}
\fi
\ifdefined\figurename
  \renewcommand*\figurename{Figure}
\else
  \newcommand\figurename{Figure}
\fi
\ifdefined\tablename
  \renewcommand*\tablename{Table}
\else
  \newcommand\tablename{Table}
\fi
}
\@ifpackageloaded{float}{}{\usepackage{float}}
\floatstyle{ruled}
\@ifundefined{c@chapter}{\newfloat{codelisting}{h}{lop}}{\newfloat{codelisting}{h}{lop}[chapter]}
\floatname{codelisting}{Listing}
\newcommand*\listoflistings{\listof{codelisting}{List of Listings}}
\makeatother
\makeatletter
\makeatother
\makeatletter
\@ifpackageloaded{caption}{}{\usepackage{caption}}
\@ifpackageloaded{subcaption}{}{\usepackage{subcaption}}
\makeatother
\usepackage{bookmark}
\IfFileExists{xurl.sty}{\usepackage{xurl}}{} % add URL line breaks if available
\urlstyle{same}
\hypersetup{
  pdftitle={Obstetricians' Knowledge of Non-Recommended pregnancy and childbirth Practices: A cross-sectional e-survey protocol},
  colorlinks=true,
  linkcolor={blue},
  filecolor={Maroon},
  citecolor={Blue},
  urlcolor={Blue},
  pdfcreator={LaTeX via pandoc}}


\title{Obstetricians' Knowledge of Non-Recommended pregnancy and
childbirth Practices: A cross-sectional e-survey protocol}
\author{}
\date{}
\begin{document}
\maketitle


\section{Study Protocol}\label{study-protocol}

\subsection{Background and rationale}\label{background-and-rationale}

Evidence-based medicine underpins the delivery of high-quality obstetric
care by promoting clinical decisions grounded in robust research
evidence, professional expertise, and patient-centered values. Despite
the availability of comprehensive evidence-based guidelines, numerous
studies have demonstrated the persistent use of interventions and
diagnostic practices in obstetrics that are not supported---or are
explicitly discouraged---by current evidence. Inadequate knowledge of
the best available research evidence can lead to unnecessary procedures,
patient distress, increased healthcare costs, and possible avoidable
complications. This can have negative repercussions on healthcare
quality and equity, particularly in Low and Middle Income Countries
(Albarqouni et al. 2023)

Assessing the level of knowledge among obstetricians regarding
non-recommended interventions is therefore essential to understanding
the factors contributing to their continued use. This survey seeks to
evaluate obstetricians' awareness and understanding of selected tests
and interventions that are inconsistent with evidence-based
recommendations. The findings are expected to highlight potential
knowledge gaps, inform targeted educational strategies, and ultimately
support the alignment of clinical practice with the best available
evidence.

\subsection{Objectives}\label{objectives}

\begin{itemize}
\item
  Primary objective: to estimate the proportion of practicing
  obstetricians who correctly identify that a given test or intervention
  is \textbf{not} recommended by evidence-based guidelines.
\item
  Secondary objectives: to identify

  \begin{itemize}
  \item
    physician characteristics associated with adequate knowledge.
  \item
    perceived barriers to guideline adherence
  \item
    preferred educational formats.
  \end{itemize}
\end{itemize}

\subsection{Study design}\label{study-design}

Observational,~cross-sectional~survey using an~anonymous~and open
e-survey. The web-based form will be distributed in January 2026 using
the secured university MS forms.

\subsection{Eligibility criteria}\label{eligibility-criteria}

\begin{itemize}
\item
  Inclusion criteria

  \begin{itemize}
  \item
    Medical doctors currently practicing as obstetricians in clinical
    practice in Egypt.
  \item
    Agree to participate (implied consent by survey completion).
  \end{itemize}
\item
  Exclusion criteria

  \begin{itemize}
  \tightlist
  \item
    Fresh graduates and house officers.
  \end{itemize}
\end{itemize}

\subsection{Sampling and recruitment}\label{sampling-and-recruitment}

\begin{itemize}
\item
  Convenience sample
\item
  Recruitment procedure: visits to hospital departments and antenatal
  care clinics to improve response rates.
\end{itemize}

\subsection{Questionnaire development}\label{questionnaire-development}

\begin{itemize}
\item
  Questionnaire sections

  \begin{itemize}
  \item
    Participant demographics and practice characteristics
  \item
    Knowledge items about specific tests or interventions
  \item
    Attitudes and barriers to guideline adherence
  \item
    Sources of clinical information and preferred educational formats
  \end{itemize}
\item
  Tests and interventions

  \begin{itemize}
  \item
    do an inherited thrombophilia evaluation for women with histories of
    pregnancy loss, fetal growth restriction (FGR), preeclampsia and
    abruption.
  \item
    screen for fetal growth restriction (FGR) with Doppler blood flow
    studies.
  \item
    use progestogens for preterm birth prevention in uncomplicated
    multifetal gestations.
  \item
    perform routine cervical length screening for preterm birth risk
    assessment in asymptomatic women before 16 weeks of gestation or
    beyond 24 weeks of gestation.
  \item
    perform antenatal testing on women with the diagnosis of gestational
    diabetes who are well controlled by diet alone and without other
    indications for testing.
  \item
    place women, even those at high-risk, on activity restriction to
    prevent preterm birth.
  \item
    perform maternal serologic studies for cytomegalovirus and
    toxoplasma as part of routine prenatal laboratory studies.
  \item
    perform serial cervical length measurement following cerclage
    placement.
  \item
    test women for methylenetetrahydrofolate reductase mutations.
  \item
    screen asymptomatic pregnant women for subclinical hypothyroidism.
  \item
    perform routine cell-free DNA screening for microdeletions.
  \item
    perform routine midtrimester serum biomarker risk stratification for
    preterm birth or preeclampsia in asymptomatic women.
  \item
    recommend delivery in a nondiabetic patient for suspected macrosomia
    before 39 0/7 weeks of gestation.
  \item
    do routine episiotomy in spontaneous vaginal births.
  \item
    do electronic fetal monitoring for low risk women in labour.
  \item
    perform umbilical artery Doppler studies as a routine screening test
    in uncomplicated pregnancies with normal fetal growth.
  \item
    do a caesarean delivery for failure of progress in labour in the
    latent phase of labour for a woman at term with a singleton fetus
    and vertex presentation.
  \item
    proceed to the early clamping (before 1 minute after birth) of the
    umbilical cord.
  \item
    schedule routine repeated cesarean section (CS) in all the pregnant
    women with a previous cesarean section.
  \end{itemize}
\end{itemize}

\subsection{Pilot and content
validation}\label{pilot-and-content-validation}

\begin{itemize}
\item
  Draft questionnaire will be reviewed by one content experts (senior
  obstetrician) and one methodologist for face validity.
\item
  Pilot testing

  \begin{itemize}
  \item
    Pilot with 5 respondents representative of target population.
  \item
    Assess completion time, item clarity, technical issues, and initial
    distribution of responses.
  \item
    Revise questionnaire and platform per pilot findings.
  \end{itemize}
\end{itemize}

\subsection{Sample size calculation}\label{sample-size-calculation}

\begin{itemize}
\tightlist
\item
  Primary outcome: proportion with correct knowledge of the
  non-recommended practices.
\end{itemize}

As no prior data were available to estimate the expected proportion, a
conservative assumption of p = 0.5 was used to maximize the required
sample size. For a 99\% confidence level (Z = 2.576) and a 10\% margin
of error (d = 0.10), the minimum calculated sample size was 167
participants. To allow for an anticipated non-response rate of 20\%, the
total number of invitations was increased to 200 participants. Dean,
Sullivan, and Soe (2013)

\subsection{Data collection
procedures}\label{data-collection-procedures}

\begin{itemize}
\item
  Implement e-survey in a secured university MS form.
\item
  Ensure anonymity: we will not collect identifiable data
\end{itemize}

\subsection{Ethical considerations}\label{ethical-considerations}

\begin{itemize}
\item
  Submit protocol to Ain Shams institutional review board (IRB).
\item
  No risk: survey of professionals knowledge with complete anonymity.
\item
  Consent: will include brief information on purpose, voluntary nature,
  and data use. Proceeding to the survey implies consent.
\item
  Data storage: store anonymous data on encrypted institutional servers.
\end{itemize}

\subsection{Consent text}\label{consent-text}

You are invited to participate in a research survey about obstetric
tests and interventions. Participation is voluntary and anonymous. The
survey takes 10 minutes. Results will be reported in aggregate only. By
proceeding you give consent for your anonymous responses to be used for
research and publications.

\subsection{Data management}\label{data-management}

\begin{itemize}
\item
  Export data to R software.
\item
  Store raw data for three years per institutional policy.
\end{itemize}

\subsection{Statistical analysis plan}\label{statistical-analysis-plan}

\begin{itemize}
\item
  Descriptive

  \begin{itemize}
  \item
    Participant characteristics: frequencies/percentages for categorical
    variables, mean ± SD or median (IQR) for continuous variables.
  \item
    Proportion correct for each knowledge item with 95\% CI.
  \item
    Total knowledge score = sum(correct items). Transform to percentage
    correct.
  \item
    Categorize knowledge ( Q1 = High, Q2 = moderate, Q3 = low, Q4 = very
    low)
  \end{itemize}
\item
  Inferential

  \begin{itemize}
  \item
    Binomial Exact test.
  \item
    Logistic regression: outcome = adequate knowledge (q1 and q2)
    (binary) to estimate adjusted odds ratios for predictors (years
    since qualification, practice type).
  \end{itemize}
\item
  Missing data

  \begin{itemize}
  \tightlist
  \item
    All fields of the e-survey will be required. This will ensure a
    complete data set.
  \end{itemize}
\item
  Statistical significance

  \begin{itemize}
  \tightlist
  \item
    Two-sided tests; p\textless0.05 considered significant.
  \end{itemize}
\item
  Software: R v4.5 (2025)
\end{itemize}

\subsection{Limitation}\label{limitation}

\begin{itemize}
\tightlist
\item
  Sampling bias due to the use of convenience sampling.
\end{itemize}

\subsection{Dissemination plan}\label{dissemination-plan}

\begin{itemize}
\item
  Publish in peer-reviewed journal and present at conferences.
\item
  We will report the study according to the CHERRIES checklist.
  (Eysenbach 2004)
\end{itemize}

\subsection{Timeline}\label{timeline}

\begin{itemize}
\item
  Week 1: questionnaire drafting
\item
  Week 2: testing and final revision
\item
  Week 3--4: Ethics approval
\item
  Week 5--6: data collection
\item
  Week 7: data analysis
\item
  Week 8: manuscript preparation and dissemination
\end{itemize}

\section*{References}\label{references}
\addcontentsline{toc}{section}{References}

\phantomsection\label{refs}
\begin{CSLReferences}{1}{0}
\bibitem[\citeproctext]{ref-albarqouni2023}
Albarqouni, Loai, Eman Abukmail, Majdeddin MohammedAli, Sewar Elejla,
Mohamed Abuelazm, Hosam Shaikhkhalil, Thanya Pathirana, et al. 2023.
{``Low-Value Surgical Procedures in Low- and Middle-Income Countries.''}
\emph{JAMA Network Open} 6 (11): e2342215.
\url{https://doi.org/10.1001/jamanetworkopen.2023.42215}.

\bibitem[\citeproctext]{ref-Dean2013OpenEpi}
Dean, A. G., K. M. Sullivan, and M. M. Soe. 2013. {``OpenEpi: Open
Source Epidemiologic Statistics for Public Health, Version 3.01.''}
\url{https://www.openepi.com}.

\bibitem[\citeproctext]{ref-Eysenbach2004}
Eysenbach, Gunther. 2004. {``Improving the Quality of Web Surveys: The
Checklist for Reporting Results of Internet E-Surveys (CHERRIES).''}
\emph{Journal of Medical Internet Research} 6 (3): e34.
\url{https://doi.org/10.2196/jmir.6.3.e34}.

\bibitem[\citeproctext]{ref-base}
R Core Team. 2025. {``R: A Language and Environment for Statistical
Computing.''} \url{https://www.R-project.org/}.

\end{CSLReferences}




\end{document}
